\documentclass{article}
\usepackage{amsmath}
%%
%% Default settings for artisynth
%%
\NeedsTeXFormat{LaTeX2e}
%%\ProvidesPackage{artisynthDoc}[2012/04/05]

\usepackage[T1]{fontenc}
\usepackage[latin1]{inputenc}
\usepackage{listings}
\usepackage{makeidx}
\usepackage{latexml}
\usepackage{graphicx}
\usepackage{framed}
\usepackage{color}

\newcommand{\pubdate}{\today}
\newcommand{\setpubdate}[1]{\renewcommand{\pubdate}{#1}}

\iflatexml
\usepackage{hyperref}
\else
%% then we are making a PDF, so include things that LaTeXML can't handle: 
%% docbook style, \RaggedRight
\usepackage{ifxetex}
\usepackage{pslatex} % fixes fonts; in particular sets a better-fitting \tt font

\usepackage[A4]{artisynth_papersize}
%\usepackage[letter]{artisynth_papersize}
\usepackage[hyperlink]{asciidoc-dblatex} 

%\usepackage{verbatim}
\usepackage{ragged2e}
\setlength{\RaggedRightRightskip}{0pt plus 4em}
\RaggedRight
\renewcommand{\DBKpubdate}{\pubdate}
\renewcommand{\DBKreleaseinfo}{}
\fi

% set hypertext links to be dark blue:
\definecolor{darkblue}{rgb}{0,0,0.8}
\definecolor{sidebar}{rgb}{0.5,0.5,0.7}
\hypersetup{colorlinks=true,urlcolor=darkblue,linkcolor=darkblue}

%%%%%%%%%%%%%%%%%%%%%%%%%%%%%%%%%%%%%%%%%%%%%%%%%%%%%%%%%%%%%%%%%%%%%%%%%%%%%
%
% Define macros for handling javadoc class and method references
%
%%%%%%%%%%%%%%%%%%%%%%%%%%%%%%%%%%%%%%%%%%%%%%%%%%%%%%%%%%%%%%%%%%%%%%%%%%%%%
\makeatletter

% code inspired by http://stackoverflow.com/questions/2457780/latex-apply-an-operation-to-every-character-in-a-string
\def\removeargs #1{\doremoveargs#1$\wholeString\unskip}
\def\doremoveargs#1#2\wholeString{\if#1$%
\else\if#1({()}\else{#1}\taketherest#2\fi\fi}
\def\taketherest#1\fi
{\fi \doremoveargs#1\wholeString}

% Note: still doesn't work properly when called on macro output ...
% i.e., \dottoslash{\concatnames{model}{base}{foo}} fails 
\def\dottoslash #1{\dodottoslash#1$\wholeString\unskip}
\def\dodottoslash#1#2\wholeString{\if#1$%
\else\if#1.{/}\else{#1}\fi\dottaketherest#2\fi}
\def\dottaketherest#1\fi{\fi \dodottoslash#1\wholeString}

% concatenates up to three class/method names together, adding '.' characters
% between them. The first and/or second argument may be empty, in which case
% the '.' is omitted. To check to see if these arguments are empty, we
% use a contruction '\if#1@@', which will return true iff #1 is empty
% (on the assumption that #1 will not contain a '@' character).
\def\concatnames
#1#2#3{\if#1@@\if#2@@#3\else #2.#3\fi\else\if#2@@#1.#3\else#1.#2.#3\fi\fi}

\newcommand{\javabase}{}
\newcommand{\setjavabase}[1]{\renewcommand{\javabase}{#1}}

\iflatexml
\newcommand{\javaclassx}[2][]{%
% Includes code to prevent an extra '.' at the front if #1 is empty. It
% works like this: if '#1' is empty, then '#1.' expands to '.', and so 
% '\if#1..' will return true, in which case we just output '#2'.
\href{@JDOCBEGIN/\concatnames{\javabase}{#1}{#2}@JDOCEND}{#2}}
\newcommand{\javaclass}[2][]{%
\href{@JDOCBEGIN/\concatnames{}{#1}{#2}@JDOCEND}{#2}}

\newcommand{\javamethodArgsx}[2][]{%
\href{@JDOCBEGIN/\concatnames{\javabase}{#1}{#2}@JDOCEND}{#2}}
\newcommand{\javamethodArgs}[2][]{%
\href{@JDOCBEGIN/\concatnames{}{#1}{#2}@JDOCEND}{#2}}
\newcommand{\javamethodAlt}[2]{%
\href{@JDOCBEGIN/\concatnames{}{}{#1}@JDOCEND}{#2}}
\newcommand{\javamethodAltx}[2]{%
\href{@JDOCBEGIN/\concatnames{\javabase}{}{#1}@JDOCEND}{#2}}

\newcommand{\javamethodNoArgsx}[2][]{%
\href{@JDOCBEGIN/\concatnames{\javabase}{#1}{#2}@JDOCEND}{\removeargs{#2}}}
\newcommand{\javamethodNoArgs}[2][]{%
\href{@JDOCBEGIN/\concatnames{}{#1}{#2}@JDOCEND}{\removeargs{#2}}}
\else
\def\javaurl{http://www.artisynth.org/doc/javadocs/}
\newcommand{\javaclassx}[2][]{{\color{darkblue}#2}}
%\href{\javaurl\dottoslash{\concatnames{\javabase}{#1}{#2}}.html}{#2}}
\newcommand{\javamethodArgsx}[2][]{{\color{darkblue}#2}}
\newcommand{\javamethodNoArgsx}[2][]{{\color{darkblue}\removeargs{#2}}}
\newcommand{\javaclass}[2][]{{\color{darkblue}#2}}
%\href{\javaurl\dottoslash{\concatnames{\javabase}{#1}{#2}}.html}{#2}}
\newcommand{\javamethodArgs}[2][]{{\color{darkblue}#2}}
\newcommand{\javamethodNoArgs}[2][]{{\color{darkblue}\removeargs{#2}}}
\newcommand{\javamethodAlt}[2]{{\color{darkblue}{#2}}}
\newcommand{\javamethodAltx}[2]{{\color{darkblue}{#2}}}
\fi

\newcommand{\javamethod}{\@ifstar\javamethodNoArgs\javamethodArgs}
\newcommand{\javamethodx}{\@ifstar\javamethodNoArgsx\javamethodArgsx}

%%%%%%%%%%%%%%%%%%%%%%%%%%%%%%%%%%%%%%%%%%%%%%%%%%%%%%%%%%%%%%%%%%%%%%%%%%%%%
%
% Define macros for sidebars
%
%%%%%%%%%%%%%%%%%%%%%%%%%%%%%%%%%%%%%%%%%%%%%%%%%%%%%%%%%%%%%%%%%%%%%%%%%%%%%

\iflatexml
\newenvironment{sideblock}{\begin{quote}}{\end{quote}}
\else
\usepackage[strict]{changepage}
\definecolor{sidebarshade}{rgb}{1.0,0.97,0.8}
\newenvironment{sideblock}{%
    \def\FrameCommand{%
    \hspace{1pt}%
    {\color{sidebar}\vrule width 2pt}%
    %{\vrule width 2pt}%
    {\color{sidebarshade}\vrule width 4pt}%
    \colorbox{sidebarshade}%
  }%
  \MakeFramed{\advance\hsize-\width\FrameRestore}%
  \noindent\hspace{-4.55pt}% disable indenting first paragraph
  \begin{adjustwidth}{}{7pt}%
  %\vspace{2pt}\vspace{2pt}%
}
{%
  \vspace{2pt}\end{adjustwidth}\endMakeFramed%
}
\fi

\iflatexml
\newenvironment{shadedregion}{%
  \definecolor{shadecolor}{rgb}{0.96,0.96,0.98}%
  \begin{shaded*}%
% Put text inside a quote to create a surrounding blockquote that
% will properly accept the color and padding attributes
  \begin{quote}%
}
{%
  \end{quote}%
  \end{shaded*}%
}
\else
\newenvironment{shadedregion}{%
  \definecolor{shadecolor}{rgb}{0.96,0.96,0.98}%
  \begin{shaded*}%
}
{%
  \end{shaded*}%
}
\fi

% Wanted to create a 'listing' environment because lstlisting is
% tedious to type and because under latexml it may need
% some massaging to get it to work properly. But hard to do
% because of the verbatim nature of listing
%\iflatexml
%\newenvironment{listing}{\begin{lstlisting}}{\end{lstlisting}}%
%\else
%\newenvironment{listing}{\begin{lstlisting}}{\end{lstlisting}}%
%\fi

\iflatexml\else
% fancyhdr was complaining that it wanted a 36pt header height ...
\setlength{\headheight}{36pt}
\fi

% abbreviation for backslash character
\newcommand\BKS{\textbackslash}


% Convenience stuff
\newcommand{\ifLaTeXMLelse}[2]{%
  \iflatexml %
  #1 %
  \else %
  #2 %
  \fi %
}

\newcommand{\ifLaTeXML}[1]{ %
  \iflatexml %
  #1 %
  \fi %
}

\makeatother


\setcounter{tocdepth}{5}
\setcounter{secnumdepth}{3}

\title{ArtiSynth Coding Standard}
\author{John Lloyd}
\setpubdate{May 9, 2012}
\iflatexml
\date{}
\fi

\begin{document}

\maketitle

\iflatexml{\large\pubdate}\fi

\tableofcontents

\section{Introduction}

It is notoriously difficult to create a coding standard that satisfies
all developers, let alone create one that is actually adhered to.
Nonetheless, there is an ArtiSynth coding standard which all
contributors are encouraged to follow.

The standard is similar in appearance to the
\href{http://java.sun.com/docs/codeconv}{coding rules provided by Sun},
although it does differ slightly. For users of Eclipse, there is an code
style file located in 
\begin{verbatim}
 $ARTISYNTH_HOME/support/eclipse/artisynthCodeFormat.xml
\end{verbatim}
that supports most of the conventions, though not all, as noted
below.

\section{Cuddle braces and indent by 3}

As with the Sun coding rules, all braces are cuddled. Basic indentation
is set to 3 to keep code from creeping too far across that page.
For example:
%
\begin{lstlisting}[]
   public static RootModel getRootModel() {
      if (myWorkspace != null) {
         return myWorkspace.getRootModel();
      }
      else {
         return null;
      }
   }
\end{lstlisting}

\section{Always used braced blocks}

As in the Sun rules, code following control constructs is always
enclosed in a braced block, even when this is not necessary. That
means that
%
\begin{lstlisting}[]
   for (int i=0; i<cnt; i++) {
      if (i % 2 == 0) {
	 System.out.println ("Even");
      }
      else {
	 System.out.println ("Odd");
      }
   }
\end{lstlisting}
should be used instead of
\begin{lstlisting}[]
   for (int i=0; i<cnt; i++)
      if (i % 2 == 0)
	 System.out.println ("Even");
      else
	 System.out.println ("Odd");
\end{lstlisting}
The reason for this is to provide greater uniformity and to
make it easier to add and remove statements from the blocks.

\section{Do not use tabs}

Code should be indented with spaces only. Tabs should not be used
since the spacing associated with tabs varies too much between coding
environments and can not always be controlled.

\section{Keep line widths to 80 columns}

Again, as with the Sun rules, code should be kept to 80 columns.  The
idea here is that it is easier to read code that doesn't creep too far
across the page. However, to maintain an 80 column width, it will
often be necessary to wrap lines.

\section{Break lines at the beginning of argument lists}

If breaking a line is necessary, the beginning of an argument list is
a good place to do it. The break should be indented by the usual 3 spaces.
The result can look at bit like Lisp:
\begin{lstlisting}[]
   public String getKeyBindings() {
      return artisynth.core.util.TextFromFile.getTextOrError (
         ArtisynthPath.getResource (
            getDirectoryName()+KEY_BINDINGS_FILE));
   }
\end{lstlisting}

If at all possible, do not break lines at the '{\tt .}' used for method or
field references. For clarity, it is better to keep these together with 
their objects. Therefore, please *do not* do this:
\begin{lstlisting}[]
   public String getKeyBindings() {
      return artisynth.core.util.TextFromFile
	 .getTextOrError (ArtisynthPath
	    .getResource (getDirectoryName()+KEY_BINDINGS_FILE));
   }
\end{lstlisting}

Note that Eclipse will not generally enforce these breaking
conventions, so you need to do this yourself.

\section{Break lines after assignment operators}

Another good place for a line break is after an assignment
operator, particularly if the left side is long:
\begin{lstlisting}[]
   LinkedList<ClipPlaneControl> myClipPlaneControls =
      new LinkedList<ClipPlaneControl>();

   if (hasMuscle) {
      ComponentList<MuscleBundle> muscles =
         ((MuscleTissue)tissue).getMuscleList();
   }
\end{lstlisting}

Again, Eclipse will not generally enforce this, so it must be done
manually.

\section{Align conditional expressions with the opening parentheses}

When line wrapping is used inside a conditional expression,
the expression itself should be aligned with the opening
parentheses, with operators placed at the right:
\begin{lstlisting}[]
         if (e.getSource() instanceof Rotator3d ||
             e.getSource() instanceof Transrotator3d ||
             e.getSource() instanceof Scaler3d) {
            centerTransform (transform, dragger.getDraggerToWorld());
         }
\end{lstlisting}
Again, note the Eclipse will not generally enforce this, and will
instead
tend to produce output like this:
\begin{lstlisting}[]
         if (e.getSource() instanceof Rotator3d
         || e.getSource() instanceof Transrotator3d
         || e.getSource() instanceof Scaler3d) {
            centerTransform (transform, dragger.getDraggerToWorld());
         }
\end{lstlisting}

\section{No space before empty argument lists}

No spaces are placed before empty argument lists, as in

\begin{lstlisting}[]
        getMainFrame().getMenuBarHandler().enableShowPlay();

        public ViewerManager getViewerManager() {
           return myViewerManager;
        }
\end{lstlisting}
This is largely to improve readability, particularly when
accessors are chained together in a single statement.

\end{document}
