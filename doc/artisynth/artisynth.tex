\documentclass{article}
%%
%% Default settings for artisynth
%%
\NeedsTeXFormat{LaTeX2e}
%%\ProvidesPackage{artisynthDoc}[2012/04/05]

\usepackage[T1]{fontenc}
\usepackage[latin1]{inputenc}
\usepackage{listings}
\usepackage{makeidx}
\usepackage{latexml}
\usepackage{graphicx}
\usepackage{framed}
\usepackage{color}

\newcommand{\pubdate}{\today}
\newcommand{\setpubdate}[1]{\renewcommand{\pubdate}{#1}}

\iflatexml
\usepackage{hyperref}
\else
%% then we are making a PDF, so include things that LaTeXML can't handle: 
%% docbook style, \RaggedRight
\usepackage{ifxetex}
\usepackage{pslatex} % fixes fonts; in particular sets a better-fitting \tt font

\usepackage[A4]{artisynth_papersize}
%\usepackage[letter]{artisynth_papersize}
\usepackage[hyperlink]{asciidoc-dblatex} 

%\usepackage{verbatim}
\usepackage{ragged2e}
\setlength{\RaggedRightRightskip}{0pt plus 4em}
\RaggedRight
\renewcommand{\DBKpubdate}{\pubdate}
\renewcommand{\DBKreleaseinfo}{}
\fi

% set hypertext links to be dark blue:
\definecolor{darkblue}{rgb}{0,0,0.8}
\definecolor{sidebar}{rgb}{0.5,0.5,0.7}
\hypersetup{colorlinks=true,urlcolor=darkblue,linkcolor=darkblue}

%%%%%%%%%%%%%%%%%%%%%%%%%%%%%%%%%%%%%%%%%%%%%%%%%%%%%%%%%%%%%%%%%%%%%%%%%%%%%
%
% Define macros for handling javadoc class and method references
%
%%%%%%%%%%%%%%%%%%%%%%%%%%%%%%%%%%%%%%%%%%%%%%%%%%%%%%%%%%%%%%%%%%%%%%%%%%%%%
\makeatletter

% code inspired by http://stackoverflow.com/questions/2457780/latex-apply-an-operation-to-every-character-in-a-string
\def\removeargs #1{\doremoveargs#1$\wholeString\unskip}
\def\doremoveargs#1#2\wholeString{\if#1$%
\else\if#1({()}\else{#1}\taketherest#2\fi\fi}
\def\taketherest#1\fi
{\fi \doremoveargs#1\wholeString}

% Note: still doesn't work properly when called on macro output ...
% i.e., \dottoslash{\concatnames{model}{base}{foo}} fails 
\def\dottoslash #1{\dodottoslash#1$\wholeString\unskip}
\def\dodottoslash#1#2\wholeString{\if#1$%
\else\if#1.{/}\else{#1}\fi\dottaketherest#2\fi}
\def\dottaketherest#1\fi{\fi \dodottoslash#1\wholeString}

% concatenates up to three class/method names together, adding '.' characters
% between them. The first and/or second argument may be empty, in which case
% the '.' is omitted. To check to see if these arguments are empty, we
% use a contruction '\if#1@@', which will return true iff #1 is empty
% (on the assumption that #1 will not contain a '@' character).
\def\concatnames
#1#2#3{\if#1@@\if#2@@#3\else #2.#3\fi\else\if#2@@#1.#3\else#1.#2.#3\fi\fi}

\newcommand{\javabase}{}
\newcommand{\setjavabase}[1]{\renewcommand{\javabase}{#1}}

\iflatexml
\newcommand{\javaclassx}[2][]{%
% Includes code to prevent an extra '.' at the front if #1 is empty. It
% works like this: if '#1' is empty, then '#1.' expands to '.', and so 
% '\if#1..' will return true, in which case we just output '#2'.
\href{@JDOCBEGIN/\concatnames{\javabase}{#1}{#2}@JDOCEND}{#2}}
\newcommand{\javaclass}[2][]{%
\href{@JDOCBEGIN/\concatnames{}{#1}{#2}@JDOCEND}{#2}}

\newcommand{\javamethodArgsx}[2][]{%
\href{@JDOCBEGIN/\concatnames{\javabase}{#1}{#2}@JDOCEND}{#2}}
\newcommand{\javamethodArgs}[2][]{%
\href{@JDOCBEGIN/\concatnames{}{#1}{#2}@JDOCEND}{#2}}
\newcommand{\javamethodAlt}[2]{%
\href{@JDOCBEGIN/\concatnames{}{}{#1}@JDOCEND}{#2}}
\newcommand{\javamethodAltx}[2]{%
\href{@JDOCBEGIN/\concatnames{\javabase}{}{#1}@JDOCEND}{#2}}

\newcommand{\javamethodNoArgsx}[2][]{%
\href{@JDOCBEGIN/\concatnames{\javabase}{#1}{#2}@JDOCEND}{\removeargs{#2}}}
\newcommand{\javamethodNoArgs}[2][]{%
\href{@JDOCBEGIN/\concatnames{}{#1}{#2}@JDOCEND}{\removeargs{#2}}}
\else
\def\javaurl{http://www.artisynth.org/doc/javadocs/}
\newcommand{\javaclassx}[2][]{{\color{darkblue}#2}}
%\href{\javaurl\dottoslash{\concatnames{\javabase}{#1}{#2}}.html}{#2}}
\newcommand{\javamethodArgsx}[2][]{{\color{darkblue}#2}}
\newcommand{\javamethodNoArgsx}[2][]{{\color{darkblue}\removeargs{#2}}}
\newcommand{\javaclass}[2][]{{\color{darkblue}#2}}
%\href{\javaurl\dottoslash{\concatnames{\javabase}{#1}{#2}}.html}{#2}}
\newcommand{\javamethodArgs}[2][]{{\color{darkblue}#2}}
\newcommand{\javamethodNoArgs}[2][]{{\color{darkblue}\removeargs{#2}}}
\newcommand{\javamethodAlt}[2]{{\color{darkblue}{#2}}}
\newcommand{\javamethodAltx}[2]{{\color{darkblue}{#2}}}
\fi

\newcommand{\javamethod}{\@ifstar\javamethodNoArgs\javamethodArgs}
\newcommand{\javamethodx}{\@ifstar\javamethodNoArgsx\javamethodArgsx}

%%%%%%%%%%%%%%%%%%%%%%%%%%%%%%%%%%%%%%%%%%%%%%%%%%%%%%%%%%%%%%%%%%%%%%%%%%%%%
%
% Define macros for sidebars
%
%%%%%%%%%%%%%%%%%%%%%%%%%%%%%%%%%%%%%%%%%%%%%%%%%%%%%%%%%%%%%%%%%%%%%%%%%%%%%

\iflatexml
\newenvironment{sideblock}{\begin{quote}}{\end{quote}}
\else
\usepackage[strict]{changepage}
\definecolor{sidebarshade}{rgb}{1.0,0.97,0.8}
\newenvironment{sideblock}{%
    \def\FrameCommand{%
    \hspace{1pt}%
    {\color{sidebar}\vrule width 2pt}%
    %{\vrule width 2pt}%
    {\color{sidebarshade}\vrule width 4pt}%
    \colorbox{sidebarshade}%
  }%
  \MakeFramed{\advance\hsize-\width\FrameRestore}%
  \noindent\hspace{-4.55pt}% disable indenting first paragraph
  \begin{adjustwidth}{}{7pt}%
  %\vspace{2pt}\vspace{2pt}%
}
{%
  \vspace{2pt}\end{adjustwidth}\endMakeFramed%
}
\fi

\iflatexml
\newenvironment{shadedregion}{%
  \definecolor{shadecolor}{rgb}{0.96,0.96,0.98}%
  \begin{shaded*}%
% Put text inside a quote to create a surrounding blockquote that
% will properly accept the color and padding attributes
  \begin{quote}%
}
{%
  \end{quote}%
  \end{shaded*}%
}
\else
\newenvironment{shadedregion}{%
  \definecolor{shadecolor}{rgb}{0.96,0.96,0.98}%
  \begin{shaded*}%
}
{%
  \end{shaded*}%
}
\fi

% Wanted to create a 'listing' environment because lstlisting is
% tedious to type and because under latexml it may need
% some massaging to get it to work properly. But hard to do
% because of the verbatim nature of listing
%\iflatexml
%\newenvironment{listing}{\begin{lstlisting}}{\end{lstlisting}}%
%\else
%\newenvironment{listing}{\begin{lstlisting}}{\end{lstlisting}}%
%\fi

\iflatexml\else
% fancyhdr was complaining that it wanted a 36pt header height ...
\setlength{\headheight}{36pt}
\fi

% abbreviation for backslash character
\newcommand\BKS{\textbackslash}


% Convenience stuff
\newcommand{\ifLaTeXMLelse}[2]{%
  \iflatexml %
  #1 %
  \else %
  #2 %
  \fi %
}

\newcommand{\ifLaTeXML}[1]{ %
  \iflatexml %
  #1 %
  \fi %
}

\makeatother


\setcounter{tocdepth}{5}
\setcounter{secnumdepth}{3}

\title{ArtiSynth Reference Manual}
\author{John Lloyd}
\iflatexml
\date{}
\fi

\begin{document}
\maketitle

\iflatexml{\large\today}\fi

\tableofcontents

Artisynth is a mechanical modeling system that allows users to combine
finite element method (FEM) components with multibody systems,
constraints, and collision handling. It is implemented in Java, and
provides a graphical interface for interactive model editing and
simulation control.

At present, most Artisynth models are created with Java code, using the
Artisynth API. The programmatic aspects of Artisynth are the focus of
this manual.

\setjavabase{artisynth.core.modelbase}

\section{Component Hierarchy}
\label{ComponentHierarchy}

\subsection{Model Components}

Artsynth models are created from a hierarchy of components. Each
component is an instance of \javaclass{ModelComponent}, which contains 
a number of methods used to maintain the component hierarchy.
These include methods for naming and numbering components:

\begin{lstlisting}
  // get the name for this component
  String getName();

  // set the name for this component
  void setName (String name);

  // get the number of this component
  int getNumber();

  // set the number of this component (for internal use only)
  void setNumber (int num);
\end{lstlisting}

Each component can be assigned a name, which can be any sequence of
characters that does not begin with a digit, does not contain the
characters '{\tt .}', '{\tt /}', '{\tt :}', '{\tt *}', or '{\tt ?}',
and does not equal the string {\tt "null"}.
For components which are not assigned a name,
\javamethod[ModelComponent]{getName()} will return {\tt null}.

Artisynth may also be configured so that components names must be
unique for all components which are children of the same parent.

Even if a component does not have a name, it has a number, which
identifies it with respect to its parent. Numbers are assigned
automatically when a component is added to its parent, and persist
unchanged until the component is removed from its parent.  This
persistence of numbers is important to ensure that components keep the
same path name as long as they are connected to the hierarchy.

Names and/or numbers can be used to form a path name for each
component that identifies its place in the hierarchy.  If a component
does not have a name, its number is used in the path instead. Some
example path names look like:

\begin{lstlisting}
  models/jawmodel/axialSprings/lat

  models/mech/models/tongue/bundles/2/elementDescs/12
\end{lstlisting}

{\tt ModelComponent} contains a number of other methods for navigating
and maintaining hierarchy structure:

\begin{lstlisting}
  // get the parent of this component
  CompositeComponent getParent();

  // sets the parent of this component (internal use only)
  void setParent(CompositeComponent parent);

  // called by the system after a component is added to a parent
  void connectToHierarchy();

  // called by the system before a component is removed from a parent
  void disconnectFromHierarchy();

  // get all hard references for this component (see below)
  void getHardReferences (List<ModelComponent> refs);

  // get all soft references for this component (see below)
  void getSoftReferences (List<ModelComponent> refs);

  // called to update the component when soft references are removed 
  void updateReferences (boolean undo, Deque<Object> undoInfo);

  // notify the parent of a change in this component
  void notifyParentOfChange (ComponentChangeEvent e);

  // returns true if this component contains state information
  boolean hasState();
\end{lstlisting}

\javamethod[ModelComponent]{getParent()} returns the component's
parent, which is a \javaclass{CompositeComponent} (Section
\ref{CompositeComponentSec}). Conversely, if {\tt getParent()} returns
null, the component is not attached to any parent and is not connected
to the hierarchy unless it is the top-level {\tt RootModel} component
(Section \ref{RootModelSec}).

When a model component is added to a parent, its method
\javamethod[ModelComponent]{connectToHierarchy()} is called, and when
it is removed from its parent, its method
\javamethod[ModelComponent]{disconnectFromHierarchy()} is called.
When either of these methods are called, the component's parent
(as returned by \javamethod[ModelComponent]{getParent()})
will be valid, and so hierarchy-dependent initialization
can be performed, like setting (or removing) back pointers to references, etc.:
\begin{lstlisting}
  connectToHierarchy () {
     ... perform hierarchy-dependent initialization ...
  }

  disconnectFromHierarchy (CompositeComponent parent) {
     ... undo hierarchy-dependent initialization ...
  }
\end{lstlisting}

The methods {\tt getHardReferences()} and {\tt getSoftReferences()}
are described in Section \ref{ComponentReferencesSec}.
 
It is also necessary to notify components in the hierarchy when there
are changes in structure or component properties, so that the
necessary adjustments can be made, including the clearing of cached
data. Notification is done using the method
\javamethod[ModelComponent]{notifyParentOfChange()}, which propagates
an appropriate change event up the hierarchy. It will typically do
this by calling the
\javamethod[CompositeComponent]{componentChanged()} method of the
parent (see Section \ref{CompositeComponentSec}).

The method \javamethod[ModelComponent]{hasState()} should return {\tt
true} if the component contains state information. This is always true
if the component contains dynamic state information such as positions
or velocities, but components may sometimes contain additional state
information (such as contact state). Structural changes
involving the addtion or removal of state-bearing components should be
announced to the system by calling {\tt notifyParentOfChange()} with a
\javaclass{StructureChangeEvent} for which
\javamethod[StructureChangeEvent]{stateIsChanged()} returns {\tt true}.

A {\tt ModelComponent} also maintains a number of flags:
\begin{lstlisting}
   // returns true if a component is selected
   boolean isSelected();

   // sets whether or not a component is selected (system use only)
   void setSelected (boolean selected);

   // returns true if a component should not be removed from its parent
   boolean isFixed();

   // sets whether or not a component should be removed from its parent
   void setFixed (boolean fixed);

   // returns true if a component is marked
   boolean isMarked();

   // sets whether or not a component is marked
   void setMarked (boolean marked);
\end{lstlisting}

All of these flags are {\tt false} by default.

\begin{description}

\item [selected] indicates whether or not a component is selected.
Components can be selected using various selection mechanisms in
the Artisynth interface, such as the navigation panel or the viewer.
When selected, its \javamethod[ModelComponent]{isSelected()} method
will return {\tt true}. 

\item [fixed] indicates, if {\tt true}, that a component should
not be removed from its parent. It is used to fix required child
components of composite components that contain otherwise removable
children (Section \ref{CompositeComponentBaseSec}).

\item [marked] is available for use by graph-processing
algorithms involving the component hierarchy, to indicate when a
component has been visited or otherwise processed. This flag should be
used with care to avoid side effects.

\end{description}

\begin{sideblock}
{\bf Important:}\\
\javamethod[ModelComponent]{setSelected()} should only be used by the
SelectionManager, and should not be called by
applications. Programmatic component selection should be performed by
calling the {\tt addSelected()} or {\tt removeSelected()} methods of the
{\tt SelectionManager}.
\end{sideblock}

Finally, all {\tt ModelComponents} implement the interface
\javaclass[maspack.util]{Scannable}, which provides methods for
writing and scanning to and from persistent storage.  Details are
given in Section \ref{writingAndScanningSec}.

For convenience, \javaclass{ModelComponentBase} provides a base
implementations of all the {\tt ModelComponent} methods.  Most
ArtiSynth components inherit from {\tt ModelComponentBase}.

\subsection{Component References}
\label{ComponentReferencesSec}

Model components can reference additional components outside of the
parent-child relationships of the hierarchy. For example, a
point-to-point spring contains two {\it references} to its end-points,
which are themselves model components. As another example, components
which implement the {\tt ExcitationComponent} interface can maintain
references to other {\tt ExcitationComponent}s to use as excitation
sources. References can be considered to be either {\it hard} or {\it
soft}. A {\tt hard} reference is one which the component requires in
order for its continued existence to be meaningful. The end-point
references for a point-to-point spring are usually hard. A {\it soft}
reference is one that the component can do without, such as the
excitation source inputs mentioned above. The methods
\javamethod[ModelComponent]{getHardReferences()} and
\javamethod[ModelComponent]{getSoftReferences()} are used to report
all hard and soft references held by a component.

\begin{sideblock}
{\bf Note:}\\
{\tt getHardReferences()} and {\tt getSoftReferences()} should report
only references held by the component itself, and {\it not} those held
by any of its descendents.
\end{sideblock}

The distinction between hard and soft references is used by the system
when components in the hierarchy are deleted. A component that holds a
hard reference to a deleted component is generally deleted as well.
However, when only soft references are deleted, then the
\javamethod[ModelComponent]{updateReferences()} method of the
referring component is called to update the component's internal
structures. {\tt updateReferences()} should also store information
about the update, to allow the update to be undone in case the method
is called later with its {\tt undo} argument set to {\tt
true}. A typical implementation pattern for {\tt updateReferences()}
is shown by the following example, in which 
{\tt maspack.util.ListRemove} is used to remove selected items from a list
of soft references, and store information needed to undo this later:
\begin{lstlisting}
   ArrayList<ModelComponent> mySoftRefs;
   ...
   void updateReferences (boolean undo, Deque<Object> undoInfo) {
      super.updateReferences (undo, undoInfo); 
      if (undo) {
         // undo the update
         Object obj = undoInfo.getFirst();
         if (obj != ModelComponentBase.NULL_OBJ) {
            ((ListRemove<ModelComponent>)obj).undo();
         }
      }
      else {
         // remove soft references which aren't in the hierarchy any more:
	 ListRemove<ModelComponent> remove = null;
         for (int i=0; i<mySoftRefs.size(); i++) {
            if (!ComponentUtils.isConnected (this, mySoftRefs.get(i)) {
	       // reference isn't in the hierarchy; request its removal
	       if (remove == null) {
                  remove = new ListRemove<ModelComponent>(mySoftRefs);
               }
	       remove.requestRemove (i);
            }
         }
         if (remove != null) {
            remove.remove();
	    undoInfo.addLast (remove);
         }
         else {
	    undoInfo.addLast (ModelComponentBase.NULL_OBJ);
         }               
      }
   }
\end{lstlisting}
When updating, the method uses
\javamethod{ComponentUtils.isConnected()} to determine which soft
references have been deleted from the hierarchy.  A {\tt ListRemove}
object is used to assemble the remove requests and then perform the
remove all at once and store information about what was removed for
possible later undoing. The remove object is appended to the end of
{\tt undoInfo}. If no undo was needed, then {\tt NULL\_OBJ} is stored
instead because {\tt Deque} objects don't accept {\tt null} arguments.
Undo information is stored at the end of the deque and removed from
the front. This allows multiple updates, including that
for the super class, to be performed in sequence.

\subsection{Composite Components}
\label{CompositeComponentSec}

\javaclass{CompositeComponent} is a subinterface of \javaclass{ModelComponent}
which can contain children. Its main methods include:

\begin{lstlisting}
  // returns the number of child components
  int numComponents();
 
  // gets a child component by name
  ModelComponent get (String name);

  // gets a child component by index
  ModelComponent get (int idx);

  // gets a child component by number
  ModelComponent getByNumber (int num);

  // returns the index of a child component
  int indexOf (ModelComponent c);

  // finds a descendent component with a specified path relative to this component
  ModelComponent findComponent (String name);

  // called when a change occurs in one of the descendants.
  void componentChanged (ComponentChangeEvent e);
\end{lstlisting}

Most of the above methods are self-explanatory. It is important to
note the difference between indices and numbers when identifying child
components. An {\it index} is the location of the child within the list of
children, starting from 0, and can change as children are added or
removed. A {\it number}, on the other hand, as described above, is
assigned automatically to a child when it is added to the parent and
persists as long as it remains.

The {\tt componentChanged()} method is called to indicate structure or
property changes. Appropriate actions may include clearing cached
data, and propogating the event further up the hierarchy (using
{\tt notifyParentOfChange()}).

\javaclass{MutableCompositeComponent} is a subinterface of
\javaclass{CompositeComponent} which allows child components to be
added and removed by an ArtiSynth application. It is a generic class
parameterized by a class type {\tt C} which must be an extension of
{\tt ModelComponent}. Its definition is:

\begin{lstlisting}
public interface MutableCompositeComponent<C extends ModelComponent>
extends CompositeComponent {

   // add a component; return false if not possible
   public boolean add (C comp);

   // add a set of components at specified index locations
   // (mainly for internal system use)
   public void addComponents (ModelComponent[] comps, int[] indices, int num);

   // remove a component; return false if not found
   public boolean remove (C comp);

   // removes a set of components and stores their original 
   // index locations (mainly for internal system use)
   public void removeComponents (ModelComponent[] comps, int[] indices, int num);
}
\end{lstlisting}

\subsection{CompositeComponentBase, ComponentList, and
ComponentListImpl}
\label{CompositeComponentBaseSec}

A default implementation of \javaclass{CompositeComponent} is provided
by \javaclass{CompositeComponentBase}. It is a non-generic class that
provides a base for composite components whose composition is created
at construction time and is not intended to change during the
running of an ArtiSynth application.

\javaclass{ComponentList} is a much more flexible class which
implements \javaclass{MutableCompositeComponent} and provides for
collections of components whose composition may be built and changed
by an application. {\tt ComponentList} is used widely to store the
many lists of components that comprise a working ArtiSynth model.

In particular, \javaclass[artisynth.core.mechmodels]{MechModel} and
\javaclass[artisynth.core.mechmodels]{FemModel3d}, the primary
ArtiSynth classes for implementing mechanical and finite element
models, are themselves subclasses of {\tt ComponentList} 
which contain lists of mechanical components (such as
particles, rigid bodies, and force effectors for {\tt MechModel},
and nodes, elements, and geometry for {\tt FemModel3d}).

In the case of {\tt MechModel}, applications can create and add their
own component lists to the model itself:
\begin{lstlisting}
   MechModel mech; 

   ....

   ComponentList<Particle> bigParticles =
      new ComponentList<Particle>(Particle.class, "big");

   ComponentList<Particle> smallParticles =
      new ComponentList<Particle>(Particle.class, "small");

   mech.add (bigParticles);
   mech.add (smallParticles);
\end{lstlisting}

\begin{sideblock}
By default, child components that belong to a
\javaclass{MutableCompositeComponent} (which includes {\tt
ComponentList}) may be selected by the ArtiSynth application for deletion.
This may be undesirable, particularly if internal structures
depend on certain child components.
Components that should not
be removed from their parents should
have their {\it fixed} flag set to {\tt true} in
the composite component constructor, either by calling
\javamethod[ModelComponent]{setFixed()},
or by adding the component using the 
\javamethod[ComponentList]{addFixed()} method of {\tt ComponentList}.
\end{sideblock}

The class \javaclass{ComponentListImpl} is available as an internal
implementation class for constructing instances of either
\javaclass{CompositeComponent} or
\javaclass{MutableCompositeComponent}.  It provides most of the
implementation methods needed for a mutable component list, which can
be exposed in the client class using delegate methods.  Components
implementing only {\tt CompositeComponent} may choose to expose only
some of these methods. For details, one should consult the source code
for \javaclass{CompositeComponentBase} or \javaclass{ComponentList}.

\setjavabase{artisynth.core.modelbase}

\section{Models}
\label{Models}

\subsection{Models and State}
\label{ModelsAndStateSec}

A \javaclass{Model} is a specific \javaclass{ModelComponent} that can
contain state and be advanced forward in time.

The methods associated with time advancement are:
\begin{lstlisting}
  // initialize the model for time t
  void initialize (double t)

  // prepare to advance the model from time t0 to t1
  StepAdjustment preadvance (double t0, double t1);

  // advance the model from time t0 to t1
  StepAdjustment advance (double t0, double t1);

  // gets the maximum step size for advancement (or -1 if undefined)
  double getMaxStepSize();
\end{lstlisting}

{\tt initialize()} is called to initialize the model for a particular
time. It is called at the beginning of a simulation (with time t = 0),
when the model is moved to a state and time defined by a
WayPoint, and when a step is repeated during adaptive stepping
(Section \ref{AdaptiveSteppingSec}).

{\tt preadvance()} is called to prepare the model for advancement from
time {\tt t0} to {\tt t1}. Often this method does nothing; it is
supplied for situtations where the model needs to perform computation
{\it before} the application of controllers or input probes (Section
\ref{ModelAgentsSec}), such as evolving internal state in some way.
The method can optionally return a \javaclass{StepAdjustment} object to
request a change in step size (Section \ref{AdaptiveSteppingSec}).

{\tt advance()} is called to advance the model from time {\tt t0} to
{\tt t1}. This is the main driver method for simulation, and typically
involves solving an ordinary differential equation (ODE) associated
with an underlying mechanical system, for which the model employs an
internal physics solver. The method can optionally return a
\javaclass{StepAdjustment} object to request a change in step size
(Section \ref{AdaptiveSteppingSec}).

A very basic simulation might proceed as follows:

\begin{lstlisting}
  t = 0;
  model.initialize (t);
  while (simulating) {
    model.preadvance (t, t+h);
    model.advance (t, t+h);
    t = t+h;
  }
\end{lstlisting}

The rate of advancement ({\tt h} in the above example) is limited by
the model's {\it effective step size}, which is nominally the maximum
step size of the root model (Section \ref{RootModelSec}). The model
can override this by providing its own maximum step size (via {\tt
getMaxStepSize()}) that is less than that of the root model.  Advance
intervals can be smaller than the effective step size, if required by
other time events imposed by WayPoints, rendering, or output
probes. The effective step size may also be reduced when adaptive
stepping is employed (Section \ref{AdaptiveSteppingSec}).

A model can contain state, which is defined to be all information
needed to deterministically advance it forward in time.

Models can contain state, as supported by the following methods:

\begin{lstlisting}
  // creates an appropriate state object for storing model state
  ComponentState createState ();

  // gets the current state for this model
  void getState (ComponentState state);

  // sets the current state for this model
  void setState (ComponentState state);
\end{lstlisting}

If a model actually maintains state, then its {\tt hasState()} method
(inherited from {\tt ModelComponent}) should return {\tt true}, and
{\tt createState()} should create an appropriate object for saving and
restoring the state using using {\tt getState()} and {\tt setState()}.

The state of a model should contain all the internal information
required to advance it forward in time. In particular, in the code
fragment,

\begin{lstlisting}
  model.getState (state);  // save state
  model.preadvance (t1, t2);
  model.advance (t1, t2);
  model.setState (state);  // restore state and
  model.initialize (t1);   // reinitialize to time t1
  model.preadvance (t1, t2);
  model.advance (t1, t2);
\end{lstlisting}

the model should have the exact same state and appearance after both
the first and seconds calls to {\tt advance()} (the call to {\tt initialize()} is
used to reset time-dependent quantities, such as time-dependent
forces). For mechanical systems, the most prominent state quantities
are the positions and velocities of the dynamic components, but there
can be other quantities as well, such as contact state and
viscoelastic state for FEM models.
  
\subsection{Model Agents}
\label{ModelAgentsSec}

As models are advanced, auxiliary agents can be employed to control
the inputs and observe the outputs of the model. These include
{\it probes}, {\it controllers}, and {\it monitors}.

\subsubsection{Probes}

A \javaclass{Probe} is an agent that sets model input data, or records
model output data, over a specific window of time. Probes that set
input data are {\it input probes} (\javaclass{InputProbe}), while those that
record output data are {\it output probes} (\javaclass{OutputProbe}). Examples
of input data include muscle excitation signals or external
forces. Output data often includes items such as velocities, postions,
or internal forces.

Input probes can be used to perform a function analagous to
the ``loading curves'' used in FEM analysis.

A probe contains several principal methods:
\begin{lstlisting}
  // apply this probe at time t
  apply (double t);

  // returns the start time of this probe
  double getStartTime();

  // sets the start time
  void setStartTime (double t);

  // returns the stop time of this probe
  double getStopTime();

  // sets the stop time
  void setStopTime (double t);

  // returns the update interval of this probe (or -1 if undefined) 
  double getUpdateInterval();

  // sets the update interval
  void setUpdateInterval (double dt);
\end{lstlisting}

The task of applying input data or recording output data is performed
in the {\tt apply()} method, which is called periodically during the time
window delimited by {\tt getStartTime()} and {\tt getStopTime()}. 

For input probes, the {\tt apply()} method is called between the the
{\tt preadvance()} and {\tt advance()} methods of the model it is
associated with (Section \ref{AssociatedModelsSec}).  For output
probes, {\tt apply()} is called after the model's {\tt advance()} method,
whenever the time advanced to equals an update time for the
probe. Update times for an output probe are given by the start and
stop times, plus any time that is an integer multiple of its {\it
effective update interval}. The effective update interval is given by
either the value returned by {\tt getUpdateInterval()}, if it is not
undefined (i.e., equal to -1), or the effective step size for the
probe's associated model (Section \ref{ModelsAndStateSec}).

Note that the start time, stop time, and update interval can also be
observed and controlled via the properties {\tt startTime}, {\tt stopTime},
and {\tt updateInterval}.

The most common types of probes used in ArtiSynth are
\javaclass{NumericInputProbe} and \javaclass{NumericOutputProbe}, which are
used to connect model properties to streams of numeric data which can
be edited and observed on the system's timeline. Numeric probe data
can also be saved to (or loaded from) external files.

\subsubsection{Controllers and monitors}

Controllers and monitors are other agents that can be used to control
or observe a simulation. Controllers are called immediately before a
model's {\tt advance()} method (and after the {\tt preadvance()}
method and the application of any input probes) and are intended to
compute control signals, while monitors are called immediately after
the {\tt advance()} method and are intended to record and process
output data.

The primary method for both is

\begin{lstlisting}
  void apply (double t0, double t1)
\end{lstlisting}

which performs the work of the agent. The times {\tt t0} and {\tt t1}
are the same times passed to the model's {\tt advance()} method.

\subsubsection{Models associated with agents}
\label{AssociatedModelsSec}

As indicated above, model agents are typically associated with a
specific model within the ArtiSynth structure, and are then applied
either before or after the {\tt advance()} method of that model.  Agents
which are not explicitly associated with a model are implicitly
associated with the root model Section \ref{RootModelSec}.

Methods to obtain and set the associated model include:
\begin{lstlisting}
  // returns the model (if any) associated with this agent
  Model getModel();

  // sets the model to be associated with this agent
  void setModel (Model model);

  // searchs for the model to be associated with this agent
  void setModelFromComponent (ModelComponent comp);
\end{lstlisting}
{\tt setModel()} sets the model directly, while {\tt setModelFromComponent()}
takes a subcomponet of a model and searches up the hierarchy to find
the model itself. For example, when connecting a component property to
a numeric probe, the system automatically determines the probe's model
by calling {\tt setModelFromComponent()} on the component hosting the
property.

\subsubsection{Model agent state}
\label{ModelAgentState}

Like models, agents can also have state, and therefore implement the
same methods {\tt hasState()}, {\tt createState()}, {\tt getState()},
and {\tt setState()} described for models in Section
\ref{ModelsAndStateSec}.

The base classes for probes, controllers, and monitors define
stateless version of these methods (i.e., {\tt hasState()} returns
{\tt false}, and {\tt getState()} and {\tt setState()} do nothing), so that agents
which actually do contain state must override these methods.

In the context of agents, state can be thought of as the internal
information that is required so that the agent's actions and effect on
its associated model are always identical for a specific time and
state.  A common example of state in the context of a controller or
monitor might be the time history used to filter a signal.

Note that the requirement ``effect on its associated model'' means
that state is also needed for input probes to handle situations when
the simulation is moved to a time and state defined by a
WayPoint. That's because probes are applied to a model only over a
specific time window, and so when the simulation is reset to a time
outside that window, it is usually necessary to reset the model
attributes controlled by the probe to their original values at that
time. As a simple example, assume that the {\tt apply()} method of an
input probe sets a value x in a model to 10 over the time window
[2,4], and that before that time, x has a value of 0. Now
if time is advanced to t = 3, x will be set to 10, and if time is then
reset to t = 1 (before the probe's window), x will need to be restored
to 0. This must be done by restoring the probe's
state, since the {\tt apply()} method will not be called at t = 1.

At present, \javaclass{NumericInputProbe} defines {\tt getState()} and
{\tt setState()} to save and restore any model property values that it
controls. No state is defined for \javaclass{NumericOutputProbe}.

\begin{sideblock}
If a controller has state, then it is important to implement
{\tt getState()} and {\tt setState()} to ensure proper behavior with
respect to both
adaptive stepping Section \ref{AdaptiveSteppingSec} and WayPoints.
If a probe or monitor has state, implementation
of {\tt getState()} and {\tt setState()} is necessary
to ensure its proper behavior with respect to WayPoints,
but not adaptive stepping, since probe and
monitor state is not changed during adaptive stepping).
\end{sideblock}

\subsection{The Root Model}
\label{RootModelSec}

All the models simulated by ArtiSynth at any given time are collected
together within a root model (\javaclass{RootModelSec}), which is the
top-level model component in the hierarchy. Every system that is
simulated by ArtiSynth is associated with a specific instance of a
{\tt RootModel}, and is typically created in code by subclassing
{\tt RootModel} and then creating and assembling the necessary components
in the subclass's constructor. Alternately, a {\tt RootModel} can be
loaded from a file, in which case a generic {\tt RootModel} is created and
then populated with structures determined from the file.

A {\tt RootModel} contains a list of all the system's models, probes,
controllers, and monitors. Methods to add or remove these include:

\begin{lstlisting}
  addModel (Model model);
  removeModel (Model model);
  removeAllModels();

  addInputProbe (InputProbe probe);
  removeInputProbe (InputProbe probe);
  removeAllInputProbes();

  addController (Controller controller);
  removeController (Controller controller);
  removeAllControllers();

  addMonitor (Monitor monitor);
  removeMonitor (Monitor monitor);
  removeAllMonitors();

  addOutputProbe (OutputProbe probe);
  removeOutputProbe (OutputProbe probe);
  removeAllOutputProbes();
\end{lstlisting}

\subsection{Advancing Models in Time}

The root model is the top-level object seen by the ArtiSynth scheduler
when running a simulation. As a simulation proceeds, the scheduler determines
the next time to advance to and then calls the root model's {\tt advance()}
method:
\begin{lstlisting}
  double t0; // current time

  while (simulating) {
     t1 = getNextAdvanceTime (t0);
     rootModel.advance (t0, t1);
  }
\end{lstlisting}

In turn, the {\tt advance()} method then individually advances all the
models contained in the root model, as described below.  The next
advance time {\tt t1}, computed by {\tt getNextAdvanceTime()}, is
determined mainly by the root model's maximum step size (returned by
{\tt RootModel.getMaxStep\-Size()}), along with other events such as
WayPoint locations and the render update rate.

The root model's maximum step size is therefore the primary simulation
step size. It can be set using the method
{\tt RootModel.setMaxStepSize()}, and is exposed as the {\tt RootModel}
property {\tt maxStepSize}.  It is also coupled to the "step" display in
the ArtiSynth GUI, and can also be obtained or set using
{\tt Main.getMaxStep()} or {\tt Main.setMaxStep()}. When a {\tt RootModel} is
created, if its maximum step size is not set explicitly (either in the
constructor or in a file specification), then it is set to a default
value which is either 0.01, or the value specified by the {\tt -maxStep}
command line argument.

A {\tt RootModel} advances each of its models in sequence, using a
procedure called {\tt advanceModel()}. Because a model may have a
maximum step size (as returned by {\tt getMaxStepSize()}) that is less
than that of the root model, or some of its output probes may have
events that preceed {\tt t1}, each model is advanced using
a series of sub-advances, with {\tt advanceModel()} taking the
form of a loop:
\begin{lstlisting}
  advanceModel (model, t0, t1) {
     ta = t0;
     while (ta < t1) {
        tb = getNextAdvanceTime (model, ta, t1);
	model.preadvance (ta, tb);
        applyInputProbes (model, tb);
        applyControllers (model, ta, tb);
        model.advance (ta, tb);
        applyMonitors (model, ta, tb);
        applyOutputProbes (model, tb);
        ta = tb;
     }
  }
\end{lstlisting}
Each time through the loop, {\tt getNextAdvanceTime()} determines the
next appropriate sub-advance time {\tt tb}, and then calls the model's
{\tt advance()} method, surrounded by the application of any probes,
controllers or monitors that are associated with it.
The {\tt preadvance()} method is called first, followed
by input probes and controllers.
Then {\tt advance()} is called, monitors are applied, and
output probes are applied if their next update time is equal to
{\tt tb}.

\begin{sideblock}
The apply time for input probes is not the time {\tt ta}
at the beginning of the time step, but rather the time {\tt tb}
corresponding to its end. This might seem
counterintuitive, but makes sense when one considers that
input probes are generally used to provide {\it targets}
for the advance process, and we typically want targets
specified for the end of the time step. If the target is a force,
then this is also consistent with implicit integration methods (used
most commonly by ArtiSynth) which solve for the system forces
at the end of the time step.
\end{sideblock}

The {\tt RootModel} {\tt advance()} method in turn calls {\tt advanceModel()}
for all models, surrounded by the application of any
probes, controllers and monitors which do not have specific models of
their own and are therefore considered to be ``owned'' by the root
model. Because some of these output probes may have event
times that preceed the desired advance time of {\tt t1}, this process
is also done in a loop:
\begin{lstlisting}
  advance (t0, t1) {
     ta = t0;
     while (ta < t1) {
        tb = getNextAdvanceTime (root, ta, t1);
        applyInputProbes (root, tb);
        applyControllers (root, ta, tb);
        for (each model m) {
           advanceModel (m, ta, tb);
        }
        applyMonitors (root, ta, tb);
        applyOutputProbes (root, tb);
        ta = tb;
     }
  }
\end{lstlisting}

Note that at present, the {\tt preadvance()} method for a root model
does nothing and is not called.

\subsection{Adaptive Stepping}
\label{AdaptiveSteppingSec}

It is possible for models to request adaptive time stepping, which may
be necessary if the model determines that a requested time step is too
large for stable simulation. The model can indicate this by having
either its {\tt preadvance()} or {\tt advance()} methods return a
\javaclass{StepAdjustment} object, which contains a recommended
scaling for the step size via its {\tt scaling} attribute.  Then, if
adaptive stepping is enabled in the root model, it will reduce the
effective time step for the model and redo the advance.  If {\tt
preadvance()} or {\tt advance()} return {\tt null}, then it is assumed
that the step size should remain unchanged (which is equivalent to
returning a {\tt StepAdjustment} with a scaling of 1).

Adaptive time stepping can be enabled or disabled using the {\tt adaptiveStepping}
property of {\tt RootModel}, or by using the {\tt RootModel} methods
\begin{lstlisting}
  boolean getAdaptiveStepping();

  void setAdaptiveStepping (boolean enable);
\end{lstlisting}

When adaptive stepping is enabled, the inner loop of the
{\tt advanceModel()} procedure described above is modified to the following:
\begin{lstlisting}
     while (ta < t1) {
        tb = getNextAdvanceTime (model, ta, s);
        model.getState (state);
        do {
	   s = GET_SCALING (model.preadvance (ta, tb));
	   if (s >= 1) {
              applyInputProbes (model, tb);
              applyControllers (model, ta, tb);
              s = GET_SCALING (model.advance (ta, tb));
           }
           if (s < 1) {
              tb = reduceAdvanceTime (model, ta, tb, s);
              model.setState (state);
              model.initialize (ta);
           }
        }
        while (s < 1);
        applyMonitors (model, ta, tb);
        applyOutputProbes (model, tb);
        ta = tb;
     }
\end{lstlisting}
where {\tt GET\_SCALING()} returns 1 if {\tt preadvance()} or
{\tt advance()} returns null, or the value of 
{\tt StepAdjustment.getScaling()} otherwise.

At the beginning of the loop, the model's state is saved in case a
retry is necessary. Then if {\tt preadvance{}} or {\tt advance()} recommend
scaling the step by s < 1, the advance time is
reduced (by reducing the model's effective step size), the state is
restored to what is was at the beginning of the step, and the step is
retried. After a step succeeds, the root model will
incrementally try to increase the effective step size, up to its
nominal value.

The exact interpretation of the scaling value s is
as follows:

\begin{description}
\item[s = 0:]
Advance unsuccessful; no recommendation as to how much to reduce the 
next step.

\item[0 < s < 1:]
Advance unsuccessful; recommend trying to reduce the step by s*(tb-ta).

\item[s = 1:]
Advance successful, no recommendation as to how much to increase the step by.

\item[s > 1:]
Advance successful; recommend trying to increase the step by s*(tb-ta).
\end{description}

When requesting a step size reduction, models may provide a string
message indicating the reason via the {\tt message} attribute of {\tt
StepAdjustment}.  The system will abort if the effective step size
falls below the minimum value specified by the {\tt RootModel}
property {\tt getMinStepSize}. At present, recommended increases in
step size are ignored and treating simply as s = 1.

\begin{sideblock}
Models are of course free to implement adaptive stepping internally, in a
way that is invisible to the root model. However, the saving and
restoring of state, along with the algorithms for step size
adjustment, are sufficiently intricate that it is generally
convenient to use the adaptive stepping provided by {\tt RootModel}.
\end{sideblock}

\section{Writing and Scanning Components}
\label{writingAndScanningSec}

\setjavabase{}

ArtiSynth model components have the ability to save and restore
themselves from persistent storage. They do this by implementing the
\javamethod[maspack.util.Scannable]{write()} and 
\javamethod[maspack.util.Scannable]{scan()} methods of
\javaclass{maspack.util.Scannable}, and the 
\javamethod[artisynth.core.modelbase.ModelComponent]{postscan()}
method of \javaclass[artisynth.core.modelbase]{ModelComponent}:

\begin{lstlisting}
  // write this component to a PrintWriter:
  void write (PrintWriter pw, NumberFormat fmt, Object ref) throws IOException

  // scan the component from a ReaderTokenizer:
  void scan (ReaderTokenizer rtok, Object ref) throws IOException

  // perform the post-scan pass for this component:
  void postscan (Deque<ScanToken> tokens, CompositeComponent ancestor);

  // normally returns true but can be overridden to return false 
  // if for some reason a component should be written to secondary storage
  boolean isWritable();
\end{lstlisting}

The operation and implementation of these methods will now be
described in detail. A summary of the key points is given in Section
\ref{summarySec}.

\subsection{Writing components}

The \javamethod[maspack.util.Scannable]{write()} method writes
information about the component to a {\tt PrintWriter}, using
\javaclass[maspack.util]{NumberFormat} to format floating point
numbers where appropriate. The {\tt ref} argument is used to provide
additional context information for generating the output, and is
specifically used to generate path names for other components
that are referenced by the component being written
(Section \ref{writingReferencesSec}).

In general, each component writes out its attributes as a list
of name/value pairs, each of the form
\begin{lstlisting}
  <name>=<value>
\end{lstlisting}
with the list itself enclosed between square brackets '{\tt [ ]}' which
serve as {\it begin} and {\it end} delimiters. The value associated with each
attribute name may itself be a quantity (such as a vector, matrix, or
another component) delimited by square brackets. For example, the
output for a \javaclass[artisynth.core.mechmodels]{Particle} component
may look like this:
\begin{lstlisting}
   [ name="primary"
     position=[ 15.0 0.0 10.0 ]
     mass=20.0
     dynamic=false
   ]
\end{lstlisting}
The output begins with an openning square bracket, followed by four
attribute/value pairs and a closing square bracket. The position
attribute is a 3-vector also enclosed between square brackets.

Within the {\tt write()} method, the above output could be produced
with code like this:
\begin{lstlisting}
  import maspack.util.*;

  void write (PrintWriter pw, NumberFormat fmt, Object ref) throws IOException {
     pw.print ("[ ");
     IndentingPrintWriter.addIndentation (2); 
     pw.println ("name=" + Write.getQuotedString(myName));
     pw.println ("position=[" + myPosition.toString (fmt) + "]");
     pw.println ("mass=" + fmt.format(myMass));
     pw.println ("dynamic=" + myDynamicP;
     IndentingPrintWriter.addIndentation (-2);
     pw.println ("]");
  }
\end{lstlisting}
Output indentation can be controlled by using an
\javaclass[maspack.util]{IndentingPrintWriter}, and
\javamethod[maspack.util]{IndentingPrintWriter.addIndentation()} 
will increase (or decrease) output indentation if
{\tt pw} is an instance of {\tt IndentingPrintWriter}.

In practice, components do not generally need to provide explicit code
to write out all their attribute values. In particular, any
information that is associated with a
\javaclass[maspack.properties]{Property} (see the Property section of
the Maspack Reference Manual) can be written out automatically using a
code fragment of the form:
\begin{lstlisting}
   getAllPropertyInfo().writeNonDefaultProps (this, pw, fmt);
\end{lstlisting}
In addition, any attribute information contained in a component's
superclass will usually be written by that superclass.
The default {\tt write()} definition for a model component
is usually looks something like this:
\begin{lstlisting}
   public void write (PrintWriter pw, NumberFormat fmt, Object ref)
   throws IOException {

      dowrite (pw, fmt, ref);
   }

   protected void dowrite (PrintWriter pw, NumberFormat fmt, Object ref)
   throws IOException {

      CompositeComponent ancestor = ComponentUtils.castRefToAncestor(ref);
      IndentingPrintWriter.printOpening (pw, "[ ");
      IndentingPrintWriter.addIndentation (pw, 2);
      writeItems (pw, fmt, ancestor);
      IndentingPrintWriter.addIndentation (pw, -2);
      pw.println ("]");
   }

   protected void writeItems (
   PrintWriter pw, NumberFormat fmt, CompositeComponent ancestor) 
   throws IOException {

      getAllPropertyInfo().writeNonDefaultProps (this, pw, fmt);
   }
\end{lstlisting}
The {\tt write()} and {\tt dowrite()} methods take care of writting
the square brackets and setting up the initial indentation. Then a
call to {\tt writeItems()} prints out all necessary property
values. The {\tt ancestor} argument obtained from {\tt ref} will be
discussed in Section \ref{writingReferencesSec}.

If all a component's attribute information is associated with
property values, then it is usually not necessary to provide
any component-specific code for writing the component: the
default implementations of {\tt write()} and {\tt writeItems()}
will handle it. If a component {\it does} have attribute information
that is not associated with a property, then it is
usually sufficient to handle this by overriding {\tt writeItems()}.
For example,
if a component has a non-property ``centroid'' attribute,
it can be written by an override of {\tt writeItems()} 
constructed like this:
\begin{lstlisting}
   protected void writeItems (
   PrintWriter pw, NumberFormat fmt, CompositeComponent ancestor) 
   throws IOException {

      super.writeItems (pw, fmt, ref);
      pw.println ("centroid=[" + getCentroid().toString (fmt) + "]");      
   }   
\end{lstlisting}

\subsubsection{Writing references}
\label{writingReferencesSec}

ArtiSynth components often contain references to other components
that are not part of their ancestor hierarchy. For example, a
two-point spring will contain
references to its two end-point particles, and a
\javaclass[artisynth.core.femmodels]{FemElement3d} will contain
references to its nodes. The set of all references refered to
by a component is returned by the combination of the component's
\javamethod[artisynth.core.modelbase.ModelComponent]{getHardReferences()}
\javamethod[artisynth.core.modelbase.ModelComponent]{getSoftReferences()}
methods.

Because they generally reside outside a component's immediate ancestor
hierarchy, information about each reference's location 
needs to be 
explicitly written and scanned as part of writing and scanning a
component. The location information is stored using the component's
path with respect to some known {\it ancestor}. This ancestor is passed to
the component's {\tt write()} method through the {\tt ref} argument, and is cast
explicitly to {\tt CompositeComponent} and passed to {\tt
writeItems()} as the {\tt ancestor} argument.  A component can
then use
\javamethod[artisynth.core.modelbase]{ComponentUtils.getWritePathName()}
to obtain the path name of each reference with respect to the
ancestor, and write this to the output.

As an example, here is a possible implementation of {\tt writeItems()}
for a two-point spring:
\begin{lstlisting}
   protected void writeItems (
   PrintWriter pw, NumberFormat fmt, CompositeComponent ancestor)
   throws IOException {

      super.writeItems (pw, fmt, ancestor);
      pw.println ("point0=" + ComponentUtils.getWritePathName(ancestor,myPnt0));
      pw.println ("point1=" + ComponentUtils.getWritePathName(ancestor,myPnt1));
   }
\end{lstlisting}
This will produce an output like this,
\begin{lstlisting}
      point0="models/points/0"
      point1="models/points/1"
\end{lstlisting}
where {\tt models/points/n} gives the path name of the n-th point with
respect to the ancestor.
Alternatively, if a component has a variable number of references,
they can be written out as a list between square brackets:
\begin{lstlisting}
   protected void writeItems (
   PrintWriter pw, NumberFormat fmt, CompositeComponent ancestor)
   throws IOException {

      super.writeItems (pw, fmt, ancestor);
      pw.println ("points=[");
      IndentingPrintWriter.addIndentation (pw, 2);
      for (ModelComponent pnt : myPnts) {
         pw.println (ComponentUtils.getWritePathName(ancestor,pnt));
      }
      IndentingPrintWriter.addIndentation (pw, -2);
      pw.println ("]");
   }
\end{lstlisting}
The above code will produce output like this:
\begin{lstlisting}
      points=[
        "models/points/0"
        "models/points/1"
        "models/points/2"
      ]
\end{lstlisting}

\begin{sideblock}
The ancestor used for reading and writing references will always be a
common ancestor of both the references and the refering component.
This may sometimes be the root model (i.e., the top of the hierarchy),
but more typically it will be the first common ancestor for which
\javamethod[artisynth.core.modelbase.CompositeComponent]
{hierarchyContainsReferences()} returns {\tt true} (implying that all
references are contained within the ancestors's descendants).  This allows
paths to be written more compactly.
Most \javaclass[artisynth.core.modelbase]{Model} components
presently enforce the {\tt hierarchyContainsReferences()} condition.
\end{sideblock}

\subsubsection{Writing child components}
\label{writingChildrenSec}

If a component is a {\tt CompositeComponent}, then it also needs to write
out its child components along with its attribute information. This
can be done by recursively calling the children's {\tt write()} methods.

If the child component configuration is fixed (i.e., the component
does not implement
\javaclass[artisynth.core.modelbase]{MutableCompositeComponent})
and the children are created in the composite's constructor,
then attribute names can be used to delimit each child. For example, suppose
a component contains two children: a list of particles and a
list of springs. This component could then be written using
a code construction such as:
\begin{lstlisting}
   protected void writeItems (
   PrintWriter pw, NumberFormat fmt, CompositeComponent ancestor) 
   throws IOException {

      super.writeItems (pw, fmt, ref);
      pw.print ("particles=");
      myParticles.write (pw, fmt, ref);
      pw.print ("springs=");
      mySprings.write (pw, fmt, ref);
   }   
\end{lstlisting}
If the composite component has been implemented internally using
an instance of 
\javaclass[artisynth.core.modelbase]{ComponentListImpl}
(Section \ref{CompositeComponentBaseSec}), then the
above can be written using the latter's
\javamethod[artisynth.core.modelbase.ComponentListImpl]{writeComponentsByName()}
method, which writes each child, using its component name as an attribute name:
\begin{lstlisting}
   ComponentListImpl myComponents; 

   protected void writeItems (
   PrintWriter pw, NumberFormat fmt, CompositeComponent ancestor) 
   throws IOException {

      super.writeItems (pw, fmt, ref);
      myComponents.writeComponentsByName (pw, fmt, ancestor);
   }   
\end{lstlisting}

One the other hand, composite components which are instances of
\javaclass[artisynth.core.modelbase]{MutableCompositeComponent} may
not have predetermined component arrangements and so these components
cannot be identified by an attribute name. Instead, the components
must simply be printed out in sequence.  For example, the {\tt
writeItems()} method for a list of particles could be programmed like
this:
\begin{lstlisting}
   protected void writeItems (
   PrintWriter pw, NumberFormat fmt, CompositeComponent ancestor) 
   throws IOException {

      super.writeItems (pw, fmt, ref);
      for (int i=0; i<particles.size(); i++) {
         particles.get(i).write (pw, fmt, ref);
      }
   }   
\end{lstlisting}
This would produce output such as:
\begin{lstlisting}
  [ position=[ 10.0 0.0 20.0 ]
    mass=20.0
  ]
  [ position=[ 5.0 0.0 10.0 ]
    mass=15.0
  ]
  [ position=[ 15.0 0.0 10.0 ]
    mass=20.0
  ]
  [ position=[ 10.0 0.0 0.0 ]
    mass=15.0
  ]
\end{lstlisting}
However, when scanning a
\javaclass[artisynth.core.modelbase]{MutableCompositeComponent}, the
{\tt scan()} method (discussed below) needs to create and scan new
child components as it encounters them in the input.  It is therefore
necessary for {\tt scan()} to know what class of component to create.
Typically, the composite component has a default component
type; for example, the default type for {\tt ComponentList<Particle>}
is \javaclass[artisynth.core.mechmodels]{Particle}. However, in some
cases the components may be subclasses of the default type, or the
default type may be an interface or abstract class and hence not
instantiable. In such instances, the output needs to be augmented with
class type information, which is placed before the openning '{\tt ['}
of the component output.  The details of how to do
this are beyond the scope of this document. However,
if the composite component has been implemented internally using
an instance of 
\javaclass[artisynth.core.modelbase]{ComponentListImpl},
then one can use the latter's
\javamethod[artisynth.core.modelbase.ComponentListImpl]{writeComponents()}
method to automatically write out all the components, along with the necessary
class information:
\begin{lstlisting}
   ComponentListImpl myComponents; 

   protected void writeItems (
   PrintWriter pw, NumberFormat fmt, CompositeComponent ancestor) 
   throws IOException {

      super.writeItems (pw, fmt, ref);
      myComponents.writeComponents (pw, fmt, ancestor);
   }
\end{lstlisting}
The result may look something like this:
\begin{lstlisting}
  [ position=[ 10.0 0.0 20.0 ]
    mass=20.0
  ]
  artisynth.core.mechmodels.SpecialParticle [
    position=[ 5.0 0.0 10.0 ]
    mass=15.0
  ]
  [ position=[ 15.0 0.0 10.0 ]
    mass=20.0
  ]
  user.projects.CustomParticle [ 
    position=[ 10.0 0.0 0.0 ]
    mass=15.0
  ]
\end{lstlisting}
In this example, the second and fourth particles have class types that differ
from the default and so class information for each is prepended
to the output.

\subsection{Scanning components}
\label{scanningComponentsSec}

The \javamethod[maspack.util.Scannable]{scan()} method reads the
component in from a token stream provided by a
\javaclass[maspack.util]{ReaderTokenizer}. This translates the input
into a stream of tokens, including words, numbers, and special token
characters (such as '{\tt [}', '{\tt ]}', and '{\tt =}'), which are
then used to parse the input. Authors implementing 
scanning code should have some familiarity with {\tt ReaderTokenizer}.
A description is beyond the scope of this document but good
documentation is available in the \javaclass[maspack.util]{ReaderTokenizer} 
class header.

The main code inside the default {\tt scan()} method for a model
component looks roughly like this:
\begin{lstlisting}
   public void scan (ReaderTokenizer rtok, Object ref) 
   throws IOException {

      rtok.scanToken ('[');
      while (rtok.nextToken() != ']') {
         rtok.pushBack();
         if (!scanItem (rtok, tokens)) {
            throw new IOException ("Unexpected token: " + rtok);
         }
      }
   }
\end{lstlisting}
The method looks for and scans the initial '{\tt [}' character (and
will throw an {\tt IOException} if this is not found). It then reads
other tokens (using
\javamethod[maspack.util.ReaderTokenizer]{nextToken()}) until the
terminating '{\tt ]}' character is found. After each token is
inspected, it is pushed back into the token stream using
\javamethod[maspack.util.ReaderTokenizer]{pushBack()} and {\tt
scanItem()} is called to try and read an individual attribute or
subcomponent from the input. If {\tt scanItem()} cannot match the
input to any attributes or child components it returns {\tt false}.

\begin{sideblock}
Note: \javaclass[maspack.util]{ReaderTokenizer} allows one
token of look-ahead, so that any read token can be pushed
back once. In particular, in the following sequence,
{\tt t1} and {\tt t2} should be the same:
\begin{verbatim}
   t1 = rtok.nextToken();
   rtok.pushBack();
   t2 = rtok.nextToken();
\end{verbatim}
\end{sideblock}

The default implementation of {\tt scanItem()} provides
code for reading property values and looks something like this:
\begin{lstlisting}
   protected boolean scanItem (ReaderTokenizer rtok, Deque<ScanToken> tokens)
   throws IOException {

      rtok.nextToken();
      // if attribute name is a property name, scan that property:
      if (ScanWriteUtils.scanProperty (rtok, this)) {
         return true;
      }
      rtok.pushBack();
      return false;
   }
\end{lstlisting}
The method begins by getting the next token and then calling
\javamethod[artisynth.core.modelbase]{ScanWriteUtils.scanProperty()}
to see if the token is a word matching the name of one of the
component's properties. If so, then {\tt scanProperty()} scans and
sets the property value and returns {\tt true}, and {\tt scanItem()}
itself returns {\tt true}. Otherwise, {\tt scanItem()} returns {\tt
false} indicating that it was unable to find a match for the input.
The {\tt tokens} argument is used to store information whose
processing must be deferred until the post-scan step, as discussed in
Section \ref{scanningReferencesSec}.

Typically, component implementations will not need to override {\tt
scan()} unless the scanning procedure calls for pre- or
post-processing, as in:
\begin{lstlisting}
   public void scan (ReaderTokenizer rtok, Object ref) 
   throws IOException {

      ... do pre-processing here ...
      super.scan (rtok, ref);
      ... do post-processing here ...
   }   
\end{lstlisting}
Note that if a class makes use of the post-scan step (Section
\ref{scanningReferencesSec}), then it may be necessary to do the
post-processing in an override of {\tt postscan()} instead.

Component implementations often {\it will} need to override {\tt
scanItem()} to scan additional attribute information. For example:
\begin{lstlisting}
  protected boolean scanItem (ReaderTokenizer rtok, Deque<ScanToken> tokens)
  throws IOException {

     rtok.nextToken();
     if (rtok.ttype == ReaderTokenizer.TT_WORD) {
        if (rtok.sval.equals ("attributeXXX") {
	   rtok.scanToken ('=');
	   ... scan information for attribute XXX ...
           return true;
        }
     }
     rtok.pushBack();
     return super.scanItem (rtok, tokens);
  }	
\end{lstlisting}
First, the method gets the next token. Then it checks if its type
({\tt ttype}) corresponds to a {\tt WORD} token and if the word's string
value ({\tt sval}) equals the attribute name {\tt attributeXXX}.  If
so, then it scans the '{\tt =}' character following attribute name,
scans whatever information is associated with the attribute, and
returns {\tt true}.  Otherwise, if no expected attribute name is
matched, the current token is pushed back, and the superclass method
is called to see if it can match the current input.

Most implementations of {\tt ModelComponent}
provide the convenience method {\tt scanAttributeName(rtok, name)}
which allows the code fragment 
\begin{lstlisting}
     if (rtok.ttype == ReaderTokenizer.TT_WORD) {        
        if (rtok.sval.equals ("attributeXXX") {
	   rtok.scanToken ('=');
\end{lstlisting}
to be replaced with
\begin{lstlisting}
      if (scanAttributeName (rtok, "attributeXXX")) {
\end{lstlisting}
Employing this in a larger example, we have
\begin{lstlisting}
  protected boolean scanItem (ReaderTokenizer rtok, Deque<ScanToken> tokens)
  throws IOException {

     rtok.nextToken();
     if (scanAttributeName (rtok, "name")) 
        myName = rtok.scanQuotedString();
        return true;
     }
     else if (scanAttributeName (rtok, "position")) {
        myPosition.scan (rtok);
        return true;
     }
     else if (scanAttributeName (rtok, "mass")) {
        myMass = rtok.scanNumber();
        return true;
     }
     else if (scanAttributeName (rtok, "dynamic")) {
        myDynamicP = rtok.scanBoolean();
        return true;
     }
     rtok.pushBack();
     return super.scanItem (rtok, tokens);
  }	
\end{lstlisting}
Here, if any of the attribute names 
{\tt name}, {\tt position}, {\tt mass}, or {\tt dynamic} are matched,
then the 
corresponding string, vector, numeric, or boolean attribute values are
scanned using
\javamethod[maspack.util.ReaderTokenizer]{scanQuotedString()}, the
vector's own {\tt scan()} method,
\javamethod[maspack.util.ReaderTokenizer]{scanNumber()}, or
\javamethod[maspack.util.ReaderTokenizer]{scanBoolean()}.  Each of
these will throw an {\tt IOException} if the input token sequence does
not match what is expected.

\subsubsection{Scanning references and post-scanning}
\label{scanningReferencesSec}

When scanning a component that contains references, the path for each
reference is used to locate the referenced component within
the component hierarchy.
However, this poses a problem: because components are
created only as the component hierarchy is recursively scanned, it is
possible that some references may not yet exist at
the time when the component is scanned. For example, if the points
referenced by a two-point spring belong to part of the hierarchy
further "to the right" of the spring components, then when the
spring is scanned the points won't yet exist and the scanning method
will be unable to find them.

The solution to this problem is to employ a two-step scanning process
in which the initial scan is followed by a secondary "post-scan" which
can be used to resolve references. Each reference path found during
the initial scan is saved for later use in the post-scan step, by
which time all components are guaranteed to have been
created. Reference information, along with any other information
needed for the post-scan step, is saved in a queue of
\javaclass[artisynth.core.util]{ScanToken}s supplied to the {\tt
scan()} method through the {\tt ref} argument. Several different types
of {\tt ScanTokens} allow different types of information to be stored:
\javaclass[artisynth.core.util]{StringToken}s are use to store
attribute names and reference paths;
\javaclass[artisynth.core.util]{ObjectToken}s are use to store object
pointers; and special marker tokens, {\tt ScanToken.BEGIN}
and {\tt ScanToken.END}, can be used as delimiters.

At a minimum, scanning each component causes {\tt BEGIN} and {\tt
END} tokens to be added to the token queue, with additional
tokens added in between as necessary.
Revisiting the basic {\tt scan()} method code
shown at the top of Section \ref{scanningComponentsSec},
we show the additional code that is needed to handle this:
\begin{lstlisting}
   public void scan (ReaderTokenizer rtok, Object ref) 
   throws IOException {

      Deque<ScanToken> tokens = (Deque<ScanToken>)ref;
      tokens.offer (ScanToken.BEGIN);
      rtok.scanToken ('[');
      while (rtok.nextToken() != ']') {
         rtok.pushBack();
         if (!scanItem (rtok, tokens)) {
            throw new IOException ("Unexpected token: " + rtok);
         }
      }
      tokens.offer (ScanToken.END);
   }
\end{lstlisting}
The token queue itself, called {\tt tokens}, is obtained from the
{\tt ref} argument via an explicit cast.
{\tt BEGIN} and {\tt END} tokens are added at the
beginning and end of the scan. In between, the token queue
is passed to {\tt scanItem()}, which adds addtional tokens
when necessary. 

Within {\tt scanItem()}, tokens are added to provide whatever
information is needed for the post-scan step.  This information is
often provided in the form of two or more tokens comprising an
attribute name/value pair, so that the post-scan step is not sensitive
to input ordering. In this sense, the information
stored in the token queue will reflect the same structure as the
tokens in the original input.

Consider the first example in \ref{writingReferencesSec}
where the reference information for a two-point spring was output as:
\begin{lstlisting}
      point0="models/points/0"
      point1="models/points/1"
\end{lstlisting}
To process this inside {\tt scanItem()}, we check for the attribute
names {\tt point0} and {\tt point1} and if either is found, we store both
the attribute name and the reference path in the token queue using
{\tt StringTokens}.  For {\tt point0}, the corresponding code
looks like
\begin{lstlisting}
      if (scanAttributeName (rtok, "point0")) {
         String refpath = rtok.scanWordOrQuotedString ('"');
	 tokens.offer (new StringToken ("point0"));
	 tokens.offer (new StringToken (refpath));
	 return true;
      }
\end{lstlisting}
Most implementations of {\tt ModelComponent}
provide the convenience method {\tt scanAndStoreReference (rtok, name, tokens)}
which allows this to be compressed into
\begin{lstlisting}
      if (scanAndStoreReference (rtok, "point0", tokens)) {
	 return true;
      }
\end{lstlisting}
One may also use
\javamethod[artisynth.core.modelbase]{ScanWriteUtils.scanAndStoreReference()}
for the same purpose.
The {\tt scanItem()} method for a two-point spring can then be written as:
\begin{lstlisting}
   protected boolean scanItem (ReaderTokenizer rtok, Deque<ScanToken> tokens)
   throws IOException {

      rtok.nextToken();
      if (scanAndStoreReference (rtok, "point0", tokens)) {
         return true;
      }
      else if (scanAndStoreReference (rtok, "point1", tokens)) {
         return true;
      }
      rtok.pushBack();
      return super.scanItem (rtok, tokens);
   }
\end{lstlisting}

Alternatively, if we have an attribute followed by 
a list of references enclosed in square brackets, such as
\begin{lstlisting}
      points=[
        "models/points/0"
        "models/points/1"
        "models/points/2"
      ]
\end{lstlisting}
then we want to store a sequence of tokens consisting of the
attribute name, a BEGIN token, the reference paths, and an END token:
\begin{lstlisting}
      "points"
      BEGIN
      "models/points/0"
      "models/points/1"
      "models/points/2"
      END
\end{lstlisting}
That can be done by a code sequence that looks like
\begin{lstlisting}
      if (scanAttributeName (rotk, "points", tokens)) {
         rtok.scanToken ('[');
         tokens.offer (new StringToken ("points"));
         tokens.offer (ScanToken.BEGIN);
         while (rtok.nextToken() != ']') {
	    if (rtok.tokenIsWordOrQuotedString ()) {
	       tokens.offer (rtok.sval);
            }
            else {
	       throw new IOException ("Error: reference path expected");
            }
         }
         tokens.offer (ScanToken.END);
         return true;
      }
\end{lstlisting}
and which is available in most ModelComponent implementations
via the convenience method {\tt scanAndStoreReferences()}:
\begin{lstlisting}
      if (scanAndStoreReferences (rtok, "point", tokens)) {
	 return true;
      }
\end{lstlisting}
One may also use
\javamethod[artisynth.core.modelbase]{ScanWriteUtils.scanAndStoreReferences()}
for the same purpose.

\subsubsection{Scanning child components}

In addition to their attributes, composite components need to scan in
their child components. This can be done by recursively calling the
children's {\tt scan()} methods.

If the child component configuration is fixed
(i.e., the component
does not implement
\javaclass[artisynth.core.modelbase]{MutableCompositeComponent}),
and the children are created in the composite's constructor
and written out using attribute names as delimiters,
then these attributes names can be used to drive the scanning.
The example composite from Section \ref{writingChildrenSec},
comprising a list of particles and a list of springs, 
could be scanned in using code such as:
\begin{lstlisting}
  protected boolean scanItem (ReaderTokenizer rtok, Deque<ScanToken> tokens)
  throws IOException {

     rtok.nextToken();
     if (scanAttributeName (rtok, "particles")) 
        tokens.offer (new ObjectToken (myParticles));
        myParticles.scan (rtok, tokens);
        return true;
     }
     else if (scanAttributeName (rtok, "springs")) {
        tokens.offer (new ObjectToken (mySprings));
        mySprings.scan (rtok);
        return true;
     }
     rtok.pushBack();     
     return super.scanItem (rtok, tokens);
  }
\end{lstlisting}
Here, an {\tt ObjectToken()} identifying each scanned componet is
stored on the token queue for later use in the post-scan step.  If the
composite component has been implemented internally using an instance
of \javaclass[artisynth.core.modelbase]{ComponentListImpl} (Section
\ref{CompositeComponentBaseSec}), then the above can be written more
succinctly using the latter's
\javamethod[artisynth.core.modelbase.ComponentListImpl]{scanAndStoreComponentByName()}
method:
\begin{lstlisting}
   ComponentListImpl myComponents; 

   protected boolean scanItem (ReaderTokenizer rtok, Deque<ScanToken> tokens)
   throws IOException {

      rtok.nextToken();
      if (myComponents.scanAndStoreComponentByName (rtok, tokens)) {
         return true;
      }
      rtok.pushBack();     
      return super.scanItem (rtok, tokens);
   }
\end{lstlisting}

On the other hand, composite components which are instances of
\javaclass[artisynth.core.modelbase]{MutableCompositeComponent} are
written out in sequence, without using attribute names but with
possible prefixed information giving information about the component's
class. When scanning in these children, {\tt scanItem()} must
determine the class for the child, create an instance of the child,
and then scan it in. The code required for these steps is beyond the
scope of this document. However, if the mutable composite
has been implemented internally using an instance of
\javaclass[artisynth.core.modelbase]{ComponentListImpl}, then one can
use the latter's methods 
\javamethod[artisynth.core.modelbase.ComponentListImpl]{scanBegin()},
\javamethod[artisynth.core.modelbase.ComponentListImpl]{scanAndStoreComponent()},
and \javamethod[artisynth.core.modelbase.ComponentListImpl]{scanEnd()}
to handle the scanning.

First, {\tt scanBegin()} is called in an override of {\tt scan()}:
\begin{lstlisting}
   ComponentListImpl myComponents; 

   public void scan(ReaderTokenizer rtok, Object ref) throws IOException {
      myComponents.scanBegin();
      super.scan (rtok, ref);
   }
\end{lstlisting}

{\tt scanAndStoreComponent()} can then be called in {\tt scanItem()} to
handle scanning of individual components:
\begin{lstlisting}
   protected boolean scanItem (ReaderTokenizer rtok, Deque<ScanToken> tokens)
   throws IOException {

      if (super.scanItem (rtok, tokens)) {
         return true;
      }
      rtok.nextToken();
      return myComponents.scanAndStoreComponent (rtok, tokens);
   }
\end{lstlisting}
The method checks to see if the input contains child component
information, and if so, scans the information, creates an instance of
the component if necessary, stores a copy of the component
in the token queue for use in post-scanning, and returns 
{\tt true}.
Note that in this case one should call {\tt scanAndStoreComponent()} {\it after}
the {\tt super} method to avoid confusing attribute names with class
information.

\begin{sideblock}
\javamethod[artisynth.core.modelbase.ComponentListImpl]{scanAndStoreComponent()}
will not create a new component if a {\it fixed} component (Section
\ref{CompositeComponentBaseSec}) of the appropriate class already
exists at the current list position. Instead, the existing component
will be scanned ``in place''.
\end{sideblock}

Finally, {\tt scanEnd()} is called in an override of {\tt postscan()}
method (discussed below):
\begin{lstlisting}
   public void postscan (
   Deque<ScanToken> tokens, CompositeComponent ancestor) throws IOException {

      super.postscan (tokens, ancestor);
      myComponents.scanEnd();
   }
\end{lstlisting}

\subsubsection{Post-scanning implementation}

Once the token queue has been built by the {\tt scan()} methods, it is
processed in the post-scan step. This is done by each component using
a {\tt postscan()} method that takes as arguments the token queue and
the ancestor with respect to which reference paths should be
evaluated. The default {\tt postscan()} method for most components
looks something like this:
\begin{lstlisting}
   public void postscan (
   Deque<ScanToken> tokens, CompositeComponent ancestor) throws IOException {

      ScanWriteUtils.postscanComponentBegin (tokens, this);
      while (tokens.peek() != ScanToken.END) {
         if (!postscanItem (tokens, ancestor)) {
            throw new IOException (
               "Unexpected token " + tokens.poll());
         }
      }
      tokens.poll(); // consume the END token
   }   
\end{lstlisting}
\javamethod[artisynth.core.modelbase.ScanWriteUtils]{postscanComponentBegin()}
gets the next token on the queue, checks that it is a {\tt BEGIN}
token, and throws an exception if this is not the case.  Then the
method simply calls {\tt postScanItem()}, which does the actual token
handling work, until a terminating {\tt END} token is found.

As is the case with {\tt scan()}, subclasses typically do not need
to override {\tt postscan()}. The exception to this is when
post-processing is required after the scan process:
\begin{lstlisting}
   public void postscan (
   Deque<ScanToken> tokens, CompositeComponent ancestor) throws IOException {

      super.postscan (tokens, ancestor);
      ... do post processing ...
   }    
\end{lstlisting}

However, any component which adds tokens in its {\tt scanItem()}
method {\it will} need to process those tokens in an override of {\tt
postscanItem()}. Tokens can be removed from the queue using
the queue's {\tt poll()} method, and can be examined (without removing
them) using the queue's {\tt peek()} method. More usefully, 
the utility class 
\javaclass[artisynth.core.modelbase]{ScanWriteUtils} provides
a number of methods for token processing, including:
\begin{lstlisting}
   boolean postscanAttributeName (tokens, name);

   C postscanReference (tokens, clazz, ancestor);

   C[] postscanReferences (tokens, refs, clazz, ancestor);
\end{lstlisting}
{\tt ModelComponentBase} also makes convenience wrappers for these
directly available within the class.
\javamethod[artisynth.core.modelbase.ScanWriteUtils]{postscanAttributeName()}
checks if the next token in the queue is a {\tt StringToken} matching
{\tt name}, and if it is, consumes that token and returns {\tt true}.
\javamethod[artisynth.core.modelbase.ScanWriteUtils]{postscanReference()}
checks that the next token is a {\tt StringToken} containing a path
reference, finds the component referenced by that path relative to
{\tt ancestor}, checks that it is an instance of {\tt clazz}, and
returns it.
\javamethod[artisynth.core.modelbase.ScanWriteUtils]{postscanReferences()}
obtains a set of component references described by a sequence of {\tt
StringTokens} located between {\tt BEGIN} and {\tt END} tokens, and
returns the referenced components in an array.  These methods will
throw an {\tt IOException} if they encounter unexpected tokens or if
referenced components cannot be found.

Employing these methods to handle the 
{\tt point0}, {\tt point1} reference example above, we obtain:
\begin{lstlisting}
   public boolean postscanItem (
   Deque<ScanToken> tokens, CompositeComponent ancestor) throws IOException {

      if (postscanAttributeName (tokens, "point0")) {
         myPnt0 = postscanReference (tokens, Point.class, ancestor);
         return true;
      }
      else if (postscanAttributeName (tokens, "point1")) {
         myPnt1 = postscanReference (tokens, Point.class, ancestor);
         return true;
      }
      return super.postscanItem (tokens, ancestor);
   }
\end{lstlisting}

Similarly, the {\tt points} reference example can be handled as:
\begin{lstlisting}
   public boolean postscanItem (
   Deque<ScanToken> tokens, CompositeComponent ancestor) throws IOException {

      if (postscanAttributeName (tokens, "points")) {
         Point[] pnts = postscanReferences (tokens, Point.class, ancestor);
         return true;
      }
      return super.postscanItem (tokens, ancestor);
   }
\end{lstlisting}

Finally, for composite components, it is necessary to call {\tt postscan()} for
each of their children. For composites implemented using
\javaclass[artisynth.core.modelbase]{ComponentListImpl},
this can be done by calling the latter's
\javamethod[artisynth.core.modelbase.ComponentListImpl]{postscanComponent()}
method. For both 
{\tt CompositeComponent} and {\tt MutableCompositeComponent}
implementations, the corresponding code looks like this:
\begin{lstlisting}
   protected boolean postscanItem (
      Deque<ScanToken> tokens, CompositeComponent ancestor) 
      throws IOException {
      
      if (myComponents.postscanComponent (tokens, ancestor)) {
         return true;
      }
      return super.postscanItem (tokens, ancestor);
   }
\end{lstlisting}
{\tt postscanComponent()} checks to see if the next token is an {\tt
ObjectToken} containing a {\tt ModelComponent}, and if it is, it
removes that token, calls the component's {\tt postscan()} method, and
returns {\tt true}.

\subsubsection{Post-scanning property values}

As described in \ref{scanningComponentsSec}, base implementations of
{\tt scanItem()} automatically read in and set property values for the
component. However, since this is done during the first scan step,
problems may arise if any properties depend on references having
already been set. An example of this is the {\tt theta} property for
the joint component
\javaclass[artisynth.core.mechmodels]{RevoluteJoint}, which requires
knowledge of the frames referenced by the joint.

The solution to this is to defer setting the values for such
properties until the post-scan step. This is done by saving the
scanned property values in the token queue, and then actually setting
the properties during the post-scan. Two convenience methods,
\javamethod[artisynth.core.modelbase]{ScanWriteUtils.scanPropertyValues()}
and 
\javamethod[artisynth.core.modelbase]{ScanWriteUtils.postscanPropertyValues()}
allow this to be done fairly easily:
\begin{lstlisting}
   // properties that must be set during post-scan:
   String[] deferredProps = new String[] { "theta", "thetaRange" };

   public boolean scanItem (ReaderTokenizer rtok, Deque<ScanToken> tokens)
   throws IOException {

      if (ScanWriteUtils.scanPropertyValues (rtok, this, deferredProps, tokens)) {
         return true;
      }
      else {
         ... remaining scanItem() implementation ...
      }
   }

   public boolean postscanItem (
   Deque<ScanToken> tokens, CompositeComponent ancestor) throws IOException {

      if (ScanWriteUtils.postscanPropertyValues (tokens, this, deferredProps)) {
         return true;
      }
      else {
         ... remaining postscanItem() implementation ...
      }
   }
\end{lstlisting}
Here, {\tt defernedProps} is an array of the names of the properties
whose values must be set in the post-scan step ({\tt "theta"} and {\tt
"thetaRange"} are taken from the specific {\tt RevoluteJoint} example).

To ensure that this works properly, it is also important that the
reference information be written out {\it before} the property
information, so that in the post-scan step, it will be set before the
property values.  That means that {\tt writeItems()} should be
structured as follows:
\begin{lstlisting}
   protected void writeItems (
   PrintWriter pw, NumberFormat fmt, CompositeComponent ancestor) 

   throws IOException {

      ... write out reference information first ...
      super.writeItems (pw, fmt, ancestor);
   }
\end{lstlisting}

\subsubsection{Invoking the complete scan process}

The two-step scanning process means that for the top-level
invocation of {\tt scan}, the application 
needs to create a token queue, call {\tt scan()} with this
queue as the {\tt ref} argument, and then call {\tt postscan()}:
\begin{lstlisting}
      ArrayDeque<ScanToken> tokens = new ArrayDeque<ScanToken>();
      comp.scan (rtok, tokens);
      comp.postscan (tokens, ancestor);
\end{lstlisting}
For convenience, the above code fragment is encapsulated into the
method \javamethod[artisynth.core.modelbase]{ScanWriteUtils.scanfull()}.

If we are scanning an entire hierarchy from scratch, then {\tt comp}
will be the root component of the hierarchy and {\tt ancestor} will
equal {\tt comp}. Otherwise, if we are scanning a new sub-hierarchy,
then {\tt comp} will be the root of the sub-hierarchy and {\tt
ancestor} may be some component higher in the existing hierarchy.

\subsection{File and token structure}
\label{fileStructureSec}

Because Artisynth components are responsible for writing and scanning
themselves, there is no mandatory file structure imposed per-se.
However, there is a structure that should be adhered to whenever
possible. Expressed loosely as a production grammar, with '*'
expressing repetition of zero or more times, this is:
\begin{lstlisting}
    component
            '[' componentItem* ']'
    
    componentItem
            attributePair
            componentSpec
    
    attributePair
            NAME '=' value
    
    componentSpec
            CLASSINFO component
            component
    
    value
            componentSpec
            list
            literal
    
    list
            '[' listItem* ']'
    
    listItem
            value
            attributePair
    
    literal
            BOOLEAN
            INTEGER
            FLOAT
            STRING
            WORD
\end{lstlisting}
Here, {\tt NAME} and {\tt WORD} are identifiers that consist of
alphanumerics, '\$', or '\_', and do not begin with a digit. {\tt
CLASSINFO} is the classname for a component, or an alias that can be
mapped to a classname using
\javamethod[artisynth.core.util]{ClassAliases.resolveClass()}.

Within an ArtiSynth file, '\#' is a comment character, causing all
remaining characters on the line to be discarded.

When tokens are saved for the post-scan step, they 
should arranged in a structure similiar to that used
for the file itself:
\begin{lstlisting}
  componentTokens
	BEGIN itemTokens* END

  itemTokens
	componentSpecTokens
	attributePairTokens

  attributePairTokens
	NAME valueTokens

  componentSpecTokens
        COMPONENT componentTokens
        componentTokens

  valueTokens
	componentSpecTokens
	listTokens
	literalTokens

  listTokens
	BEGIN <listItemTokens>* END

  listItemTokens
	valueTokens
	attributePairTokens

  literalTokens
	STRING
	OBJECT
\end{lstlisting}
Here, {\tt BEGIN} and {\tt END} are {\tt
ScanToken.BEGIN} and {\tt ScanToken.END}, {\tt NAME} is a
\javaclass[artisynth.core.util]{StringToken} with an attribute name as
a value, {\tt COMPONENT} is a
\javaclass[artisynth.core.util]{ObjectToken} with a reference to the
object as a value, and {\tt STRING} and {\tt OBJECT} are
\javaclass[artisynth.core.util]{StringToken} and
\javaclass[artisynth.core.util]{ObjectToken}, respectively.

\subsection{Debugging}

Debugging write and scan methods is generally not too difficult
because of the ascii nature of the data files. A good first test is to
write components out, read them back in, and then write them out a
second time and make sure that the second output equals the
first. Scan methods will generally throw {\tt IOException}s when
unexpected input is encountered, and these usually provide the
offending line number. 

Problems that occur in post-scan can be slightly harder to solve
because the token queue is not normally written out in any place where it can
be inspected. To help with this, once can use
\javamethod[artisynth.core.modelbase]{ScanWriteUtils.setTokenPrinting()}
to enable the token queue produced by
\javamethod[artisynth.core.modelbase]{ScanWriteUtils.scanfull()} to be
printed to the standard output. It is also possible to print a token
queue directly using 
\javamethod[artisynth.core.modelbase]{ScanWriteUtils.printTokens()}.

\subsection{Summary}
\label{summarySec}

The main points concerning component writing and scanning are
as follows:

\begin{enumerate}

\item Writing and scanning are done using the component's
\javamethod[maspack.util.Scannable]{write()},
\javamethod[maspack.util.Scannable]{scan()},
\javamethod[artisynth.core.modelbase.ModelComponent]{postscan()}
methods.  These methods usually employ {\tt writeItems()}, {\tt
scanItem()}, and {\tt postscanItem()} to handle the writing and
scanning of individual attributes and child components.

\item Where possible, the structure described in Section
\ref{fileStructureSec} should be adhered to. 

\item Scanning is a two-step process, involving a scan step and a
post-scan step. This is to accomodate the fact that some aspects of
scanning (most importantly the evaluation of references) cannot be
done until the entire component hierarchy has been constructed.
Information needed for the post-scan step (such as reference path
names) should be stored in the token queue passed to {\tt scan()} and
{\tt scanItems()}.

\item It is usually only necessary for a component implementation to
override {\tt writeItems()}, {\tt scanItem()}, and {\tt
postscanItem()}. Property values are usually written and scanned
automatically by the base implementations of {\tt writeItems()} and
{\tt scanItem()}. If a component does not contain references or
non-property attributes, it may not be necessary for the
implementation to override any methods at all.

\item Composite components need to call {\tt
write()}, {\tt scan()}, and {\tt postscan()} for their child
components. Composites implemented using
\javaclass[artisynth.core.modelbase]{ComponentListImpl}
can do this using methods supplied by that class, such as
{\tt writeComponents()}, {\tt scanAndStoreComponent()}, 
and {\tt postscanComponent()}.

\item A complete scan operation involves creating
a token queue and then calling both {\tt scan()} and
{\tt postscan()} for the top-level component.
This can be done using the convenience method
\javamethod[artisynth.core.modelbase]{ScanWriteUtils.scanfull()}.

\item The utility class
\javaclass[artisynth.core.modelbase]{ScanWriteUtils} contains a large
number of methods that facilitate writing and scanning.

\end{enumerate}

\end{document}


